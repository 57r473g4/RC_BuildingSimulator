% !TEX root = ../Thesis.tex

\chapter{Methodology}\
\label{ch:methodology}

Code for a basic RC model was written. Different Solving approaches were compared.
The 5R1C model from ISO was implemented by PJ, and combined with the ASF model.
#Potential Comparison with 1r1c model and 3r 1c mdoel

Simulating multiple building archetypes: CEA database
Simulation parameters: 
- CEA database provides 15 archetypes of which 11 are used: MULTI_RES SINGLE_RES HOTEL OFFICE RETAIL FOODSTORE RESTAURANT INDUSTRIAL SCHOOL HOSPITAL GYM 
- For each archetype:
	-building properties for six distinct periods are provided: 1920 1920-1970 1970-1980 1980-2005 2005-2020 2020-2030. In addition, two sets of properties are defined: one for new construction and another for renovation.
		possibility to first study new buildings, and later expand to older buildings
	- heating and cooling setpoints and setbacks (energy_minimization.py)
	- probability of occupancy, lighting, and electrical appliances on given weekdays and weekends ## need to figure out how to use them

- Room geometry parameters:
	- different glazing ratios/room dimensions ## check limits in ISO standard

- Weather file and orientation:
	- Zurich S, Zurich W
	- Dubai S, Dubai W

- Systems analysis: analyse the effectiveness of different heating and cooling systems coupled with the ASF



